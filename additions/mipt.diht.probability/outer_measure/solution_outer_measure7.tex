\begin{task}{7}
Построить такие неизмеримые относительно классической меры Лебега на $[0;1]$ множества $A_1, A_2$, что $A_1 \cup A_2$ измеримо.
\end{task}
\begin{solution}
Возьмем в качестве $A_1$ -- множество Витали на $[0;1]$, $A_2$ - его дополнение. Множество Витали не измеримо. Пусть его дополнение измеримо. Но тогда множество Витали было бы измеримым. Следовательно, $A_2$ неизмеримо. $A_1 \cup A_2 = E = [0;1]$, то есть их объединение измеримо.

\begin{definition}
Множество Витали - неизмеримое по Лебегу множество. Его построение:

Рассмотрим такое отношение эквивалентности $\sim$: $x \sim y$, если $x - y \in \mathbb{Q}$. Это отношение разбивает $[0;1]$ на классы эквивалентности. Выберем по представителю в каждом классе. Полученное множество $A$ будет неизмеримым.
\end{definition}

\begin{proposition}
Множество Витали неизмеримо.
\end{proposition}
\begin{proof}
Занумеруем все рациональные числа на $[-1;1]$. Получим следующую последовательность $\{r_n\}_{n=1}^{\infty}$. Пусть $A_n = A + r_n$. Докажем, что полученные множества не пересекаются. Пусть это не так. Тогда 
\begin{equation}
    \exists x = a_n + r_n = a_m + r_m, n \not = m.
\end{equation}
Но тогда $a_m - a_n \in \mathbb{Q}$, а значит $a_n$ и $a_m$ лежат в одном множестве. Противоречие.

Пусть в $A_n$ находится измеримое множество $C_n$ с мерой $d > 0$. Тогда
\begin{equation}
    \forall m \exists C_m \subseteq A_m: \mu(C_m) = d.
\end{equation}
Но 

\begin{equation}
    \bigsqcup_{n=1}^{\infty}{C_n} \subseteq \bigsqcup_{n=1}^{\infty}{A_n} \subseteq [-1; 2],
\end{equation}

откуда

\begin{equation}
    \sum_{n=1}^{\infty}{\mu(C_n)} = \sum_{n=1}^{\infty}{d} \leq 3.
\end{equation}

Противоречие.

С другой стороны,

\begin{equation}
    [0;1] \subseteq \bigsqcup_{n=1}^{\infty}{A_n}.
\end{equation}

Если мера $A_n$ равна нулю, то и сумма тоже будет равна нулю. Но $\mu([0;1]) = 1$. Противоречие.

Следовательно, множество Витали неизмеримо.

\end{proof}

\end{solution}
